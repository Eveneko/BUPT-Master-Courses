% !TEX TS-program = xelatex
% !TEX encoding = UTF-8
% !Mode:: "TeX:UTF-8"

\documentclass[onecolumn,oneside]{BUPTHomework}

\author{胡玉斌}
\sid{2021111054}
\title{作业1}
\coursecode{}
\coursename{网络空间安全学科论文写作指导}

\begin{document}
  \maketitle

  \section*{如何写好论文的每个部分?}

  学术论文的核心三要素是原创性、科学性、完整性。它们构成了论文的三大论据类型。原创性就是创新性,体现在新发现、新方法、新技术上。科学性就是正确性,体现在基于目前人类的认知水平在逻辑推理上的对错。正确性在人类进步的历史上是相对的,即过去被认为是正确的理论,将来可能变成错误的理论,反之亦然。完整性就是深广性,体现在所论证的内容在参数关系或元素间相互作用上究竟是一个孤立的个别现象,还是一个具有普遍意义的广泛现象。

  原创性和完整性代表科研工作的意义和水平。实际上,完整性是原创性和科学性的基础,因为从完整性分析上能够识别原创性贡献,而且深度和广度代表正确性的一部分,即以偏概全是错误的。既原创又完整的成果,意义最为重大。原创但不完整的成果,具有发表的价值,但是需要后人加以完善。完整但非原创的成果,通常属于综述,也具有发表的价值,便于同行学习。既非原创又不完整的成果,没有发表的价值,这种科研工作需要按照科学方法回炉重做,才能合格。

  除了语言和格式问题之外,原创性、科学性、完整性是论文审稿专家用来评判论文质量的主要标准。因此,论点和论据必须围绕这三个方面构造和展开。简而言之,就是需要论证所提出的观点为什么是新颖而重要的,为什么是正确的,以及为什么在深度和广度上是完整的。如果能够使用强有力的论据令人无懈可击地说清楚这三个问题,就是成功的论文。

  这三个问题并非需要在论文中设立三节分别论述。实际上,它们是穿插在论文的各部分中。学术论文具有固定的格式,通常包括摘要、引言、材料和方法、结果、讨论、结论这六个核心部分。其中,摘要基本就是后五个部分的高度浓缩提炼。论点需要在摘要、引言和结论中提出。论据需要分布在引言、材料和方法、结果、讨论这四个部分中。关于原创性的意义,通常在引言中介绍完问题的背景后予以论述。关于科学性的论据,需要出现在材料和方法、结果(包括公式推导和数据)中。

  需要在引言中提出论点,在结论中总结论点,在结果和讨论中按照逻辑推理顺序给出理由和证据,并且避免在段与段之间以及上下句之间出现逻辑断裂和重复啰嗦。在学术论文的三大论据的安置上,需要遵循以下三个原则:

  1.将原创性的论据无重复地布置在引言、材料和方法、结果、讨论这四部分中。
  2.将科学性的论据无重复地布置在材料和方法、结果这两部分中。
  3.将完整性的论据根据系统工程的元素分类方法布置在材料和方法、结果、讨论这三部分中,覆盖深度和广度。

  在我们采集完数据,想清楚研究问题之后开始撰写论文。通常的逻辑是:Title, Abstract, Introduction, Materials \& Methods, Results, Discussion, Conclusion(s). 

  \subsection*{Title and Abstract}

  文章的题目和摘要通常是被读者阅读最多的部分。 不但要非常短小精炼,更要保证读者阅读完题目和摘要已经能清楚文章的研究内容,主要结论和创新点。

  \subsection*{Introduction}

  引言,顾名思义,就是讲个简短的故事告诉读者这篇文章为什么会出现。通常包括:

  \begin{itemize}
    \item 研究的主题以及它在科学上或者工程上为什么重要。
    \item 总结别人的工作,提炼还未解决的问题。
    \item 提出你的文章准备要研究的具体问题 – 通常短文章重点聚焦1-2个问题.,长文章2-3个问题。围绕少数主要问题讨论可以让整个文章逻辑性强,论点突出,也可以有更多的篇幅进行深度的讨论分析。
    \item 最后可以再次强调文章的主要结论和意义。
  \end{itemize}

  \subsection*{Materials/method}

  该部分需要详细描述材料的制备过程,实验的具体过程及详细的参数。目的是让感兴趣的读者或者同行能够重复。这需要实验过程中保持良好的记录习惯,注重细节参数。

  实验过程的描述最好按照时间顺序,有逻辑的进行描述,时态通常是过去时。另外,很多工作可能有相同或者相似的实验方法,具体的描述还不能复制别人的内容,可以进行改写。

  \subsection*{Result}

  \begin{itemize}
    \item  利用副标题,段落对结果进行有逻辑的描述
    \item 包含主要图片和表格 – 次要的支撑性的数据放在补充材料;简洁为主,图片组合避免拥挤等
    \item 主要结果要紧紧围绕在introduction中提出的科学问题
  \end{itemize}

  \subsection*{Discussion}

  讨论是体现文章深度的部分,是告诉读者这个研究为什么重要的最后机会。通常聚焦在一下几个问题:

  \begin{itemize}
    \item 提炼文章的主要结果和意义
    \item 和同行的工作进行比较,强调创新点
    \item 提出自己的理论模型,解释文章中新的实验结果
    \item 提出自己的理论模型,解释文章中新的实验结果
    \item 提出后续的问题等
  \end{itemize}

  \subsection*{Conclusion}

  \begin{itemize}
    \item 总结实验主要实验结果和理论解释
    \item 提炼文章的创新点
    \item 强调文章的意义/重要性
  \end{itemize}

\end{document}
