% !TEX TS-program = xelatex
% !TEX encoding = UTF-8
% !Mode:: "TeX:UTF-8"

\documentclass[onecolumn,oneside]{BUPTHomework}

\author{胡玉斌}
\sid{2021111054}
\title{作业2}
\coursecode{}
\coursename{网络空间安全学科论文写作指导}

\begin{document}
  \maketitle

  \section*{你认为科研道德方面有哪些坚决不能触碰的红线?}

  古人云:人无信不立,业无信不兴。科研诚信是每一位科研工作者必须坚守的底线,各位老师同学们作为科研活动的参与者,要遵守学术规范、坚守学术诚信、完善学术人格、维护学术尊严,预防学术不端行为的发生,让我们共同营造诚实守信、追求真理、崇尚创新、鼓励探索、永攀高峰的良好氛围。

  什么是学术不端?

  2016年教育部发布的《高等学校预防与处理学术不端行为办法》第二十七条中对学术不端作出了严格规定:经调查,确认被举报人在科学研究及相关活动中有下列行为之一的,应当认定为构成学术不端行为:

  \begin{itemize}
    \item 剽窃、抄袭、侵占他人学术成果;
    \item 篡改他人研究成果;
    \item 伪造科研数据、资料、文献、注释,或者捏造事实、编造虚假研究成果;
    \item 未参加研究或创作而在研究成果、学术论文上署名,未经他人许可而不当使用他人署名,虚构合作者共同署名,或者多人共同完成研究而在成果中未注明他人工作、贡献;
    \item 在申报课题、成果、奖励和职务评审评定、申请学位等过程中提供虚假学术信息;
    \item 买卖论文、由他人代写或者为他人代写论文;
    \item 其他根据高等学校或者有关学术组织、相关科研管理机构制定的规则,属于学术不端的行为。
  \end{itemize}

\end{document}
