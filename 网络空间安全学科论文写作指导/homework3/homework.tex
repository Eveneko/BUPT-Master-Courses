% !TEX TS-program = xelatex
% !TEX encoding = UTF-8
% !Mode:: "TeX:UTF-8"

\documentclass[onecolumn,oneside]{BUPTHomework}

\author{胡玉斌}
\sid{2021111054}
\title{作业3}
\coursecode{}
\coursename{网络空间安全学科论文写作指导}

\begin{document}
  \maketitle

  \section*{如何写好学位毕业论文的每个部分? }

  \subsection*{学位论文格式}
  \begin{enumerate}
    \item 封面
    \item 内封(扉页)
    \item 声明
    \item 摘要
    \item 目录
    \item 符号说明
    \item 论文正文
    \item 参考文献
    \item 附录
    \item 致谢
    \item 攻读学位期间发表论文
  \end{enumerate}

  \subsection*{标题}

  标题长度一般不要超过20字。需要鲜明,具有高度概括性和表意的准确性。

  \subsection*{关键词}

  一般为3-8个,词性为名词,主要是专业术语,选取专业性较强的词汇。

  \subsection*{摘要}

  要求通俗易懂,尽量避免行话,是全文的简介,尽可能简短。一般需要对写作背景、研究思路、研究方法、研究结论进行叙述。

  \subsection*{正文}

  为了做到层次分明、脉络清晰,常常将正文部分分成几个大的段落。这些段落即所谓逻辑段,一个逻辑段可包含几个小逻辑段,一个小逻辑段可包含一个或几个自然段,使正文形成若干层次。论文的层次不宜过多,一般 不超过五级。

  \subsection*{参考文献}

  为了反映文章的科学依据、作者尊重他人研究成果的严肃态度以及向读者提供有关信息的出处,正文之后一般应列出参考文献表。

  引文应以原始文献和第一手资料为原则。所有引用别人的观点或文字,无论曾否发表,无论是纸质或电子版,都必须注明出处或加以注释。凡转引文献资料,应如实说明。

  对已有学术成果的介绍、评论、引用和注释,应力求客观、公允、准确。伪注、伪造、篡改文献和数据等,均属学术不端行为。

  \subsection*{致谢}

  一项科研成果或技术创新,往往不是独自一人可以完成的,还需要各方面的人力,财力,物力的支持和帮助。因此,在许多论文的末尾都列有"致谢"。主要对论文完成期间得到的帮助表示感谢,这是学术界谦逊和有礼貌的一种表现。

\end{document}
