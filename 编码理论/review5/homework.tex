% !TEX TS-program = xelatex
% !TEX encoding = UTF-8
% !Mode:: "TeX:UTF-8"

\documentclass[onecolumn,oneside]{BUPTHomework}

\usepackage{blindtext}

\author{胡玉斌}
\sid{202111054}
\title{Review\ 第五章\ RM码}
\coursecode{3131100063}
\coursename{编码理论}

\begin{document}
  \maketitle
  
  \section*{线性码性质回顾}

  \subsection*{定义1}

  一个码长为$n$的$p$元码$C$叫做线性码,
  是指$C$是向量空间$F^n_p$的向量子空间,
  即$C$满足如下的性质:对$F_p$中任意元素$\alpha$和$\beta$,
  如果$c_1$和$c_2$属于$C$,
  则$\alpha c_1+\beta c_2$也属于$C$。

  \subsection*{定理1}

  设$C$是参数为$[n,k]$的$p$元线性码。

  (1) 若$G$是$C$的一个生成矩阵,
  而$H$是$F_p$上一个$(n-k)$行$n$列的矩阵,
  则$H$是$C$的一个校验矩阵当且仅当$rank(H)=n-k$,
  并且$HG^T=0_{n-k,k}$

  (2) 若$G=(I_k,P)$,$H=(=P^T,I_{n-k})$,
  则$G$是$C$的一个生成矩阵上去仅当$H$是$C$的一个校验矩阵

  \subsection*{定义2}

  设$C$是一个$p$元线性码,参数为$n,k,d$。
  从$C$是$F^n_p$的一个$k$维向量子空间。
  考虑$F^n_p$中的如下子集和:

  $$C'=\{a\in F^n_p \vert \forall\ c \in C, (a,c)=0 \}$$

  即$C'$是与$C$中所有码字都正交的那些向量组成的集合。
  称$C'$为$C$的对偶码。

  \subsection*{定理2}

  若$C$是参数为$[n,k]$的$p$元线性码。
  则$C'$也是$p$元线性码,
  码长和信息位数分别为$n$和$n-k$。

  如果$C \in C'$,称$C$为自正交码。
  如果$C = C'$,称$C$为自对偶码。

  \section*{RM码的定义}

  \subsection*{定义3}

  设$m$为正整数。
  一个$m$元布尔函数$f=f(x_1,...,x_m)$是由$F^m_2$到$F_2$的映射,
  即$m$个变量$x_1,...,x_m$均取值于$F_2$,
  并且函数值也属于$F_2$。

  由于$F^m_2$中向量的个数为$2^m$,
  而$f$在每个向量的取值均彼此独立地可取1或0,
  所以$m$元布尔函数共有$2^{2^m}个$。

  \subsection*{定理3}

  每个$m$元布尔函数$g(x_1,...,x_m)$均可唯一地表示为$g(x_1,...,x_m)=c+c_1x_1+..+c_mx_m+c_{12}x_1x_2+c_{13}x_1x_3+...+c_{m-1,m}x_{m-1}x_m+c_{123}x_1x_2x_3+...+c_{12...m}x_1x_2...x_m$(其中所有系数和常数都属于$F_2$)

  \subsection*{定义4}

  设$m \geq 1,n=2^m,0 \leq r \leq m$。
  向量空间$F^n_2$的子集合
  
  $$RM(r,m)=\{c_f=(f(v_0),f(v_1),...,f(v_{n-1})) \in F^n_2 \vert f \in B_m, deg(f) \leq r \}$$

  叫做$r$阶的Reed-Muller码(简称RM码)。这里$v_i \in F^m_2$

  \subsection*{定理4}

  设$m \geq 1,f \in B_m$,当$r \leq m-1$时,$w(c_f)$为偶数。

  \subsection*{定理5}

  $RM$码$RM(r,m)$是线性码,基本参数为$[n,k,d]=[2^m,\sum^r_{t=0}(^m_t),2^{m-r}]$。

  \subsection*{定理6}

  当$0 \leq m-1$时,$RM(r,m)$的对偶码为$RM(m-r-1.m)$。

  \section*{RM码的编译码}

  \begin{enumerate}
    \item RM码的生成矩阵
    \item RM码的校验矩阵
    \item RM码编译码实例
  \end{enumerate}

  \subsection*{心得体会\&建议}

  第一次翻转课堂,同学准备用心,ppt也很棒,上台讲课的同学很卖力。美中不足的是,内容有些许枯燥,例子不够充分,证明不够明确,给我们后面的同学也是一种提醒。

\end{document}
