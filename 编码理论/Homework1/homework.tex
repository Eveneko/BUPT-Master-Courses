% !TEX TS-program = xelatex
% !TEX encoding = UTF-8
% !Mode:: "TeX:UTF-8"

\documentclass[onecolumn,oneside]{BUPTHomework}

\usepackage{blindtext}

\author{胡玉斌}
\sid{202111054}
\title{第一章\ 作业题}
\coursecode{3131100063}
\coursename{编码理论}

\begin{document}
  \maketitle
  
  \section*{5.}
  考虑表1中列出的信源符号和它们相应的概率。对于这个码,求信源的熵、每个符号的平均二元字节数、该码的效率。

  \begin{table}[!htbp]
  \centering
  \begin{tabular}{|c|c|c|c|}
  \hline
  符号    & 概率   & 自信息    & 码字 \\ \hline
  $x_1$ & 0.40 & 1.3219 & 1  \\ \hline
  $x_2$ & 0.35 & 1.5146 & 00 \\ \hline
  $x_3$ & 0.25 & 2.0000 & 01 \\ \hline
  \end{tabular}
  \caption{信源符号和它们相应的概率、自信息及码字}
  \end{table}

  \section*{12.}
  为什么要用信道编码器?信道编码要满足什么定理?请给出该定理的内容。

  \section*{18.}
  对于如图2所示的BSC信道,信源符号发生的概率为 。求:
  \begin{enumerate}
    \item 信源 $X$ 中事件 $x_1$ 和 $x_2$ 分别的自信息(以比特为单位);
    \item 接收符号 $y_i(i=1,2)$ 发生的概率;
    \item 求条件概率 $P(x_i \vert y_i)$ ;
    \item 收到消息 $y_i(i=1,2)$ 后,获得的关于 $x_i(i=1,2)$ 的信息量;
    \item 信源 $X$ 和信源 $Y$ 的信息熵;
    \item 条件熵 $H(X \vert Y)$ 和 $H(Y \vert X)$。
  \end{enumerate}

  \section*{19.}
  上机题。
  \begin{enumerate}
    \item 如何编程序实现霍夫曼编码?
  \end{enumerate}

\end{document}
