% !TEX TS-program = xelatex
% !TEX encoding = UTF-8
% !Mode:: "TeX:UTF-8"

\documentclass[onecolumn,oneside]{BUPTHomework}

\usepackage{blindtext}

\author{胡玉斌}
\sid{2021111054}
\title{Review\ 第十四章\ 网络编码}
\coursecode{3131100063}
\coursename{编码理论}

\begin{document}
  \maketitle
  
  \section*{p18}

  简单介绍了网络编码的概述之后,我们将介绍网络编码相关基础知识,这部分知识将涉及一些图论中网络流的概念。

  \section*{p19}
  
  在网络编码中,它能够实现理论上的最大传输容量。
  从这幅图我们可以看到这样几个消息,这是一张有向图,图中边权为正,源点是1,终点是6。
  这里就存在一个最大流的问题,那什么是最大流。

  \section*{p20}

  把源点比作工厂的话,最大流问题就是求从工厂最大可以发出多少货物,是不至于超过道路的容量限制。
  采用Ford-Fulkerson Algorithm算法可以计算网络中的最大流。

  \section*{p21}

  在介绍FFA算法之前,我们先介绍一些网路流基础知识。
  首先,请分清楚 网络(或者流网络,Flow Network)与 网络流(Flow)的概念。

  网络是指一个有向图,
  每条边都有一个权值c,称之为容量。
  其中有两个特殊的点:源点s和汇点t。

  \section*{p22}

  现在我们设f(u,v) 满足以下特性

  \begin{itemize}
    \item 容量限制:对于每条边,流经该边的流量不得超过该边的容量
    \item 斜对称性:每条边的流量与其相反边的流量之和为 0
    \item 流守恒性:从源点流出的流量等于汇点流入的流量
  \end{itemize}

  那么 f 称为网络 G 的流函数,f(u,v)称为边的流量。
  c(u,v)-f(u,v) 称为边的剩余容量

  我们有一张图,要求从源点流向汇点的最大流量(可以有很多条路到达汇点),就是我们的最大流问题,对应的算法是 Ford-Fulkerson 增广路算法。

  \section*{p23}

  在后续算法介绍中,我们会提到一个名词,残量网络。
  我们已经介绍了剩余流量。

  对于流函数 f,残存网络 $G_f$(Residual Network)是网络 G 中所有结点和剩余容量大于 0 的边构成的子图。

  \section*{p24}

  刚刚提到FFA算法是一个增广路算法,简单介绍一下在图G中的增广路。

  在原图 G 中若一条从源点到汇点的路径上所有边的 剩余容量都大于 0,这条路被称为增广路。

  或者说,在残存网络 $G_f$ 中,一条从源点到汇点的路径被称为增广路。
  
\end{document}
