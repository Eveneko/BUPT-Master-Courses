% !TEX TS-program = xelatex
% !TEX encoding = UTF-8
% !Mode:: "TeX:UTF-8"

\documentclass[onecolumn,oneside]{BUPTHomework}

\usepackage{blindtext}

\author{胡玉斌}
\sid{2021111054}
\title{Review\ 第六章\ 卷积码}
\coursecode{3131100063}
\coursename{编码理论}

\begin{document}
  \maketitle
  
  \section*{卷积码的发展历史}

  1955年,P.Elias 首次提出卷积码的概念

  \section*{卷积码与分组码的区别}

  两者都是纠错码。

  \subsection*{卷积码}
  \begin{itemize}
    \item 序列逻辑电路
    \item 有记忆性
  \end{itemize}

  \subsection*{分组码}
  \begin{itemize}
    \item 组合逻辑电路
    \item 无记忆性
  \end{itemize}

  \subsection*{对比}

  \begin{itemize}
    \item 在同样的编码效率下,卷积码的性能优于分组码
    \item 在同样的纠错能力下,卷积码的实现比分组码简单
  \end{itemize}

  \section*{离散序列的非循环卷积运算}

  \subsection*{定义}

  设 $f(n)$和$g(n)$是两个序列,则序列$f$和$g$的离散卷积运算为 $f*g m$

  \section*{二元(2,1,2)卷积码}

  \subsection*{编码器}

  该编码器主要由$m=2$级位移寄存器,$n=2$个模2加法器组成 (线性前馈位移寄存器)

  \subsection*{冲激响应}

  通过令$u=(100...)$所得到的两个输出序列

  \subsection*{输出序列}

  \begin{itemize}
    \item 信息序列$u=(u_1,u_2,u_3,...)$每次进入编码器1比特
    \item 编码器的两个输出序列
  \end{itemize}

  \subsection*{码字}

  输出序列分别为 $v^1\ v^1_0,v^1_1,v^1_2...,v^2\ v^2_0,v^2_1,v^2_2\ ...$的$(2,1,m)$的卷积码的码字
  为$V^{(1)}$和$V^{(2)}$的交错。
  
  \subsection*{编码方程}

  以$u=(u_1,u_2,u_3,...)$为输入,
  以$v^1\ v^1_0,v^1_1,v^1_2...,v^2\ v^2_0,v^2_1,v^2_2\ ...$为生成序列
  的$(2,1,m)$的卷积码的输出序列分别由方程$V^{(1)}$和$V^{(2)}$得到

  \subsection*{生成矩阵}

  $(2,1,m)$的卷积码的的生成矩阵:将生成序列$g^{1}$和$g^{2}$交织后形成的半无限矩阵。

  \section*{二元(2,1,3)卷积码}

  \section*{二元(3,2,1)卷积码}

  \section*{二元(3,2,m)卷积码}

  \section*{(n,k,m)卷积码}

  \section*{心得体会\&建议}

  \begin{enumerate}
    \item 讲课效果特别差,读稿机器人(第一位同学)
    \item slides配色不合适,公式几乎看不清(第二位同学)
  \end{enumerate}
  
\end{document}
