% !TEX TS-program = xelatex
% !TEX encoding = UTF-8
% !Mode:: "TeX:UTF-8"

\documentclass[onecolumn,oneside]{BUPTHomework}

\author{胡玉斌}
\sid{2021111054}
\title{第二次作业}
\coursecode{3131100788}
\coursename{物联网安全}

\begin{document}
  \maketitle
  
随着区块链和物联网的广泛接受,技术世界似乎再次融合。
物联网的主要目的是将物理世界与数字世界连接起来。
随着物联网渗透到我们的设备中,当数十亿个设备通过中央通信通道连接时,数据安全问题可能是一个重大挑战。
数据被泄露的机会越来越高,区块链就是这个问题等解决方案。

区块链技术的核心优势是验证成本和联网成本。
验证成本与区块链网络的能力有关,由于采用了分散式验证共识方法,区块链网络能够以符合成本效益的方式验证交易。
区块链是去中心化分类账本设置,其中整个网络在加密关系下以去中心化方式连接。
一旦输入到分类帐中的条目就无法更改,从而使整个网络没有单点故障和数据泄漏。
区块链技术正在赋予匿名性,并为使用它的网络提供抵抗黑客攻击的能力。
它可以帮助弥合物联网和数据安全之间的鸿沟,并使物联网网络更加安全和不变。
区块链技术将使IoT设备数据比以往任何时候都更加私密。

然而区块链上的智能合约也会存在一些漏洞,开发者也在不断修复代码。
我们希望可以利用这部分信息来修复这些漏洞。
我们提出了一种通过比较历史版本来寻找EOSIO平台智能合约的安全问题的方法。
在分析了EOSIO智能合约中的所有安全问题后,我们将整合了安全问题的原因、模式和解决方案。
最终,我们将探索如何自动修复智能合约中的漏洞。

\end{document}
